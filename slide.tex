\documentclass{beamer}
% Copyright 2015 by Do Phan Thuan

% Loại mẫu slice
%\usetheme{AnnArbor}
%\usetheme{Antibes}
\usetheme{Boadilla}
%\usetheme{CambridgeUS}
%\usetheme{Hannover}

% Ký tự tiếng Việt
\usepackage[utf8]{vietnam}
% Chèn ảnh
\usepackage{graphicx}
% Chèn đường dẫn 
\usepackage{url}

% Vẽ đồ thị

% Insert code
\usepackage{listings}
\lstset{language=C++,
   %keywords={break,case,catch,continue,else,elseif,end,for,function,
   %   global,if,otherwise,persistent,return,switch,try,while},
   basicstyle=\ttfamily,
   keywordstyle=\color{blue},
   commentstyle=\color{red},
   stringstyle=\color{dkgreen},
   frame=lrtb,
   %frame=5 pt,
   numbers=left,
   numberstyle=\tiny\color{gray},
   stepnumber=1,
   numbersep=10pt,
   backgroundcolor=\color{white},
   tabsize=4,
   showspaces=false,
   showstringspaces=false}
% Tô mầu cho bảng
\usepackage{colortbl}


\usepackage{color}

\definecolor{dkgreen}{rgb}{0,0.6,0}
\definecolor{gray}{rgb}{0.5,0.5,0.5}
\definecolor{mauve}{rgb}{0.58,0,0.82}
  
\definecolor{Xanh}{rgb}{0,0.5,1}
\definecolor{Do}{rgb}{1,0.25,0}
\definecolor{Vang}{rgb}{1,1,0}
\definecolor{Datroi}{rgb}{0,0,1}
% Vẽ hình
\usepackage{tikz}
\usetikzlibrary{arrows,shapes}
% Vẽ mạch điện
\usepackage[siunitx,european resistors]{circuitikz}

% multirow
\usepackage{multirow}

\usepackage{pbox}

% Tô mầu cho bảng
\usepackage{colortbl}
\definecolor{Xanh}{rgb}{0,0.5,1}
\definecolor{Do}{rgb}{1,0.25,0}
\definecolor{Vang}{rgb}{1,1,0}
\definecolor{Datroi}{rgb}{0,0,1}

% Một vài ký hiệu thường dùng
\def\R{{\mathbb R}}
\def\N{{\mathbb N}}
\def\X{{\mathcal X}}
\def\Y{{\mathcal Y}}
\def\F{{\mathcal F}}
\def\P{{\mathcal P}}
\def\E{{\mathbb E}}
\def\I{{\mathbb I}}
\def\sign{{\rm sign}}

% Xác định khoảng dãn trong bảng
%\renewcommand\arraystretch{1.6}

% a few macros
\newcommand{\bi}{\begin{itemize}}
\newcommand{\ei}{\end{itemize}}
\newcommand{\ig}{\includegraphics}
\newcommand{\subt}[1]{{\footnotesize \color{subtitle} {#1}}}

% named colors
\definecolor{offwhite}{RGB}{249,242,215}
\definecolor{foreground}{RGB}{255,255,255}
\definecolor{background}{RGB}{24,24,24}
\definecolor{title}{RGB}{107,174,214}
\definecolor{gray}{RGB}{155,155,155}
\definecolor{subtitle}{RGB}{102,255,204}
\definecolor{hilight}{RGB}{22,155,104}
\definecolor{vhilight}{RGB}{255,111,207}
\definecolor{lolight}{RGB}{155,155,155}
%\definecolor{green}{RGB}{125,250,125}

% Minted
%\usepackage{minted}
%\usemintedstyle{monokai}
%\newminted{cpp}{fontsize=\footnotesize}

% Graph styles
\tikzstyle{vertex}=[circle,fill=black!50,minimum size=15pt,inner sep=0pt, font=\small]
\tikzstyle{selected vertex} = [vertex, fill=red!24]
\tikzstyle{edge} = [draw,thick,-]
\tikzstyle{dedge} = [draw,thick,->]
\tikzstyle{weight} = [font=\scriptsize,pos=0.5]
\tikzstyle{selected edge} = [draw,line width=2pt,-,red!50]
\tikzstyle{ignored edge} = [draw,line width=5pt,-,black!20]

%gets rid of bottom navigation bars
\setbeamertemplate{footline}[frame number]{}

%gets rid of bottom navigation symbols
%\setbeamertemplate{navigation symbols}{}

%gets rid of footer
%will override 'frame number' instruction above
%comment out to revert to previous/default definitions
%\setbeamertemplate{footline}{}

% Tác giả, Tiêu đề, vân vân
\title[]{{\huge \bf Một vài thuật toán sắp xếp trên mô hình song song} \\}
\author[]{
Nguyễn Tuấn Đạt\\
Đặng Quang Trung\\
}

\institute[]{
%\inst{1}% 
}

\logo{\includegraphics[scale=0.05]{hust.jpg} \vspace{220pt}}

\begin{document}

\begin{frame}
\titlepage
\end{frame}

\begin{frame}{Nội dung}
\tableofcontents
\end{frame}
\section{Giới thiệu MPI}
\begin{frame}{Giới thiệu MPI}
\bi
\item MPI là một thủ viện chuẩn của trao đổi thông điệp giữa nhiều máy tính chạy một chương trình song song trên bộ nhớ phân tán.
\item Nó cho phép tính toán song song trên các clusters và các mạng không đồng nhất.
\item Được thiết kế cho phép (mở) phát triển các thư viện phần mền song song.
\item Được thiết kế để cung cấp quyền truy cập vào phần cứng song song cho 
\bi
\item Người dùng cuối.
\item Người viết thư viện.
\item Người phát triển tool.
\ei
\ei
\end{frame}
\section{Các thuật toán sử dụng}
\begin{frame}{Merge Sort}
\bi
\item \textbf{Ý tưởng:} Xây dựng một cây xử lý
\bi
\item Số lượng nút lá của cây bằng số lượng bộ xử lý.
\item Chiều cao của cây $log(p)$ (p là số bộ xử lí).
\ei
\item Mỗi nút lá chứa danh sách các phần tử. Áp dụng giải thuật sắp xếp tuần tự cho mỗi nút lá.
\item Kết quả ở mỗi nút lá sẽ được cung cấp về các nút cha(quá trình trộn 2 danh sách để được 1 danh sách mới).Các nút cha lại gửi tiếp.
\item Cuối cùng nút gốc sẽ là sư hòa trộn thành dãy đã được sắp xếp.
\ei
\end{frame}
\begin{frame}{Xây dựng cây}
\begin{figure}[H]
\includegraphics[scale=0.6]{img1.png}
\caption{Cây xử lý độ cao 3}
\end{figure}
\end{frame}
\begin{frame}{Trộn kết quả các nút}
\bi
\item Mỗi nút trong cây xử lí 1 quá trình riêng
\item Ý tưởng trộn mượn ý tưởng binary heap khi nó thực hiện trong 1 mảng với gốc là 0.
\item Với mỗi phần tử trong mảng có chỉ số con k có con trái là 2*k + 1 và con phải là 2*k + 2 và phần tử cha là $\frac{k-1}{2}$.
\ei
$$\bigcirc(n)+\bigcirc(n/p*\log p)+\bigcirc(n/p \log (n/p))$$
\end{frame}
\begin{frame}{Trộn kết quả các nút}
\begin{figure}[H]
\includegraphics[scale=0.4]{img2.png}
\caption{Cây xử lý độ cao 3}
\end{figure}
\end{frame}
\begin{frame}{Odd-Even}
\textbf{Ý tưởng:} Sử dụng biến thể của odd-even để mở rộng cho nhiều bộ xử lí
\bi
\item Chia dữ cho các bộ xử lí (p).
\item Mỗi giai đoạn:(1 $\longrightarrow$ p):
\bi
\item Sắp xếp dư liệu địa phương trên mỗi bô
\item Tìm các đối tác của bộ xử lý dựa trên giai đoạn và rank của chúng.
\item Nếu bộ xử lí có đối tác:
\bi
\item[- ] gửi dữ liệu đến đối tác của chúng.
\item[- ] nhận dữ liệu từ đối tác của chúng.
\item[- ] Nếu rank của chúng nhỏ, thì giữ lại các phần tử nhỏ nhất(dữ liệu của bộ xử lí + dư liệu của đối tác).
\item[- ] Trái lại thì giữ lại các phần tử lớn nhất(dữ liệu của bộ xử lí + dữ liệu của đối tác).
\ei
\ei
\ei

\end{frame}
\begin{frame}{QuickSort}
Ý tưởng: Dựa trên nguyên lý chia để trị.\\
	Xét hệ thống có p bộ xử lý.\\
	Thuật toán hoạt động như sau:
	Tại mỗi bước thuật toán cố gắng chia dữ liệu thành 2 group dựa vào một key chọn sẵn.
	\begin{enumerate}
		\item Bước 1( i lần): Group j sẽ được chia làm 2 phần dựa key (key được họn ngẫu nhiên từ tiến trình chủ trong group). $i \in [1..log(p)]$
		\item Bước 2: Sử dụng thuật toán sắp xếp tối ưu để sắp xếp tại mỗi tiến trình.
		\item Bước 3: Gửi mảng đã sắp xếp về tiến trình root.
\end{enumerate}
$$\bigcirc(n)+\bigcirc(n/p*\log p)+\bigcirc(n/p \log (n/p))$$
\end{frame}
\begin{frame}{Mô tả}
\begin{figure}[H]
\includegraphics[scale=0.3]{quicksort.jpg}
\caption{Hoạt động của thuật toán quicksort}
\end{figure}
\end{frame}
\begin{frame}{SampleSort}
Ý tưởng: Dựa trên ý tưởng của thuật toán bucketSort thuật toán sampleSort sẽ chọn ra một tập mẫu để quy đinh các bucket cho bucketSort.\\
Hoạt động:
\begin{itemize}
\item Bước 1: Sắp xếp mảng cục bộ tại mỗi tiến trình bằng thuật  toán tối ưu.
\item Bước 2: Tại mỗi tiến trình chọn ra p-1 số cách đều trong dữ liệu đã sắp xếp và gửi về tiến trình root.
\item Bước 3: Tiến trình root tổng hợp và chọn ra p-1 số mẫu để quy định khoảng cho các bucket và gửi dữ liệu này đến mỗi tiền trình.
\item Bước 4: Mỗi tiến trình chia dữ liệu vào các bucket và gửi dữ liệu ấy đến tiến trình ứng với bucket đó. 
\item Bước 5: Mỗi tiến trình sắp xếp dữ liệu tại bucket của mình.
\item Bước 6: Mỗi tiến trình gửi dữ liệu đã sắp xếp về tiến trình root. 
\end{itemize}
{\color{hilight} Độ phức tạp}
 $$\bigcirc(n/p \log(n/p) ) +\bigcirc(p^2 log p) + \bigcirc(p \log (n/p)) + \bigcirc(n/p)+ \bigcirc(p \log p)  $$
\end{frame}
\begin{frame}{Mô tả}
\begin{figure}[H]
\includegraphics[scale=0.4]{SS.png}
\caption{Hoạt động của thuật toán sample sort}
\end{figure}
\end{frame}
\section{Kết quả}
\subsection{Thực hiện}
\begin{frame}{Thực hiện}
\begin{itemize}
\item Thực hiện trên 2 máy bộ xử lý core i5.
\item Dữ liệu sử dụng 5 bộ với n=1000,10000,100000,1000000,1000000.
\item Thực hiện đo 10 lần với mỗi trường hợp sau đó tính trung bình.
\end{itemize}

\end{frame}


\subsection{Kết quả}
\begin{frame}{Biểu đồ  với số lượng bộ xử lý }
\begin{figure}[H]
\includegraphics[scale=0.4]{ms.png}
\includegraphics[scale=0.4]{ov.png}
\caption{Biểu đồ trên số lượng bộ xử lý}
\end{figure}


\end{frame}
\begin{frame}{Biểu đồ  với số lượng bộ xử lý}
\begin{figure}[H]
\includegraphics[scale=0.4]{qs.png}
\caption{Biểu đồ trên số lượng bộ xử lý}
\end{figure}
\end{frame}
\begin{frame}{Biểu đồ speedup}
\begin{figure}[H]
\includegraphics[scale=0.4]{sp.png}
\caption{Biểu đồ speedup}
\end{figure}

\end{frame}
\begin{frame}{Biểu đồ thể hiện thời gian truyền}
\begin{figure}[H]
\includegraphics[scale=0.4]{tpame.png}
\includegraphics[scale=0.4]{tpaov.png}
\caption{Biểu đồ thời gian truyền}
\end{figure}


\end{frame}
% TODO: Book
\begin{frame}{Tài liệu tham khảo}
\section*{Tài liệu tham khảo}
-http://parallelcomp.uw.hu/ch09lev1sec4.html\\
-http://parallelcomp.uw.hu/ch09lev1sec5.html\\
-Slide bài giảng lập trình song song thầy Nguyễn Tuấn Dũng

\end{frame}
\end{document}