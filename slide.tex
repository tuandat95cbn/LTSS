\documentclass{beamer}
% Copyright 2015 by Do Phan Thuan

% Loại mẫu slice
%\usetheme{AnnArbor}
%\usetheme{Antibes}
\usetheme{Boadilla}
%\usetheme{CambridgeUS}
%\usetheme{Hannover}

% Ký tự tiếng Việt
\usepackage[utf8]{vietnam}
% Chèn ảnh
\usepackage{graphicx}
% Chèn đường dẫn 
\usepackage{url}

% Vẽ đồ thị

% Insert code
\usepackage{listings}
\lstset{language=C++,
   %keywords={break,case,catch,continue,else,elseif,end,for,function,
   %   global,if,otherwise,persistent,return,switch,try,while},
   basicstyle=\ttfamily,
   keywordstyle=\color{blue},
   commentstyle=\color{red},
   stringstyle=\color{dkgreen},
   frame=lrtb,
   %frame=5 pt,
   numbers=left,
   numberstyle=\tiny\color{gray},
   stepnumber=1,
   numbersep=10pt,
   backgroundcolor=\color{white},
   tabsize=4,
   showspaces=false,
   showstringspaces=false}
% Tô mầu cho bảng
\usepackage{colortbl}


\usepackage{color}

\definecolor{dkgreen}{rgb}{0,0.6,0}
\definecolor{gray}{rgb}{0.5,0.5,0.5}
\definecolor{mauve}{rgb}{0.58,0,0.82}
  
\definecolor{Xanh}{rgb}{0,0.5,1}
\definecolor{Do}{rgb}{1,0.25,0}
\definecolor{Vang}{rgb}{1,1,0}
\definecolor{Datroi}{rgb}{0,0,1}
% Vẽ hình
\usepackage{tikz}
\usetikzlibrary{arrows,shapes}
% Vẽ mạch điện
\usepackage[siunitx,european resistors]{circuitikz}

% multirow
\usepackage{multirow}

\usepackage{pbox}

% Tô mầu cho bảng
\usepackage{colortbl}
\definecolor{Xanh}{rgb}{0,0.5,1}
\definecolor{Do}{rgb}{1,0.25,0}
\definecolor{Vang}{rgb}{1,1,0}
\definecolor{Datroi}{rgb}{0,0,1}

% Một vài ký hiệu thường dùng
\def\R{{\mathbb R}}
\def\N{{\mathbb N}}
\def\X{{\mathcal X}}
\def\Y{{\mathcal Y}}
\def\F{{\mathcal F}}
\def\P{{\mathcal P}}
\def\E{{\mathbb E}}
\def\I{{\mathbb I}}
\def\sign{{\rm sign}}

% Xác định khoảng dãn trong bảng
%\renewcommand\arraystretch{1.6}

% a few macros
\newcommand{\bi}{\begin{itemize}}
\newcommand{\ei}{\end{itemize}}
\newcommand{\ig}{\includegraphics}
\newcommand{\subt}[1]{{\footnotesize \color{subtitle} {#1}}}

% named colors
\definecolor{offwhite}{RGB}{249,242,215}
\definecolor{foreground}{RGB}{255,255,255}
\definecolor{background}{RGB}{24,24,24}
\definecolor{title}{RGB}{107,174,214}
\definecolor{gray}{RGB}{155,155,155}
\definecolor{subtitle}{RGB}{102,255,204}
\definecolor{hilight}{RGB}{22,155,104}
\definecolor{vhilight}{RGB}{255,111,207}
\definecolor{lolight}{RGB}{155,155,155}
%\definecolor{green}{RGB}{125,250,125}

% Minted
%\usepackage{minted}
%\usemintedstyle{monokai}
%\newminted{cpp}{fontsize=\footnotesize}

% Graph styles
\tikzstyle{vertex}=[circle,fill=black!50,minimum size=15pt,inner sep=0pt, font=\small]
\tikzstyle{selected vertex} = [vertex, fill=red!24]
\tikzstyle{edge} = [draw,thick,-]
\tikzstyle{dedge} = [draw,thick,->]
\tikzstyle{weight} = [font=\scriptsize,pos=0.5]
\tikzstyle{selected edge} = [draw,line width=2pt,-,red!50]
\tikzstyle{ignored edge} = [draw,line width=5pt,-,black!20]

%gets rid of bottom navigation bars
\setbeamertemplate{footline}[frame number]{}

%gets rid of bottom navigation symbols
%\setbeamertemplate{navigation symbols}{}

%gets rid of footer
%will override 'frame number' instruction above
%comment out to revert to previous/default definitions
%\setbeamertemplate{footline}{}

% Tác giả, Tiêu đề, vân vân
\title[]{{\huge \bf Một vài thuật toán sắp xếp trên mô hình song song} \\}
\author[]{
Nguyễn Tuấn Đạt\\
Đặng Quang Trung\\
}

\institute[]{
%\inst{1}% 
}

\logo{\includegraphics[scale=0.05]{hust.jpg} \vspace{220pt}}

\begin{document}

\begin{frame}
\titlepage
\end{frame}

\begin{frame}{Nội dung}
\tableofcontents
\end{frame}
\section{Giới thiệu MPI}
\begin{frame}{Giới thiệu MPI}
\bi
\item MPI là một thủ viện chuẩn của trao đổi thông điệp giữa nhiều máy tính chạy một chương trình song song trên bộ nhớ phân tán.
\item Nó cho phép tính toán song song trên các clusters và các mạng không đồng nhất.
\item Được thiết kế cho phép (mở) phát triển các thư viện phần mền song song.
\item Được thiết kế để cung cấp quyền truy cập vào phần cứng song song cho 
\bi
\item Người dùng cuối.
\item Người viết thư viện.
\item Người phát triển tool.
\ei
\ei
\end{frame}
\section{Các thuật toán sử dụng}
\begin{frame}{Meger Sort}
\bi
\item \textbf{Ý tưởng:} Xây dựng một cây xử lý
\bi
\item Số lượng nút lá của cây bằng số lượng bộ xử lý.
\item Chiều cao của cây $log(p)$ (p là số bộ xử lí).
\ei
\item Mỗi nút lá chứa danh sách các phần tử. Áp dụng giải thuật sắp xếp tuần tự cho mỗi nút lá.
\item 
\item Kết quả ở mỗi nút lá sẽ được cung cấp về các nút cha(quá trình trộn 2 danh sách để được 1 danh sách mới).Các nút cha lại gửi tiếp.
\item Cuối cùng nút gốc sẽ là sư hòa trộn thành dãy đã được sắp xếp.
\ei
\end{frame}
\begin{frame}{Xây dựng cây}
\begin{figure}[H]
\includegraphics[scale=0.6]{img1.png}
\caption{Cây xử lý độ cao 3}
\end{figure}
\end{frame}
\begin{frame}{Trộn kết quả các nút}
\bi
\item Mỗi nút trong cây xử lí 1 quá trình riêng
\item Ý tưởng trộn mượn ý tưởng binary heap khi nó thực hiện trong 1 mảng với gốc là 0.
\item Với mỗi phần tử trong mảng có chỉ số con k có con trái là 2*k + 1 và con phải là 2*k + 2 và phần tử cha là $\frac{k-1}{2}$.
\ei
\end{frame}
\begin{frame}{Trộn kết quả các nút}
\begin{figure}[H]
\includegraphics[scale=0.4]{img2.png}
\caption{Cây xử lý độ cao 3}
\end{figure}
\end{frame}
\begin{frame}{Odd-Even}
\textbf{Ý tưởng:} Sử dụng biến thể của odd-even để mở rộng cho nhiều bộ xử lí
\bi
\item Chia dữ cho các bộ xử lí (p).
\item[1.] Mỗi giai đoạn:(1 $\longrightarrow$ p):
\bi
\item Sắp xếp dư liệu địa phương trên mỗi bô
\item Tìm các đối tác của bộ xử lý dựa trên giai đoạn và rank của chúng.
\item Nếu bộ xử lí có đối tác:
\bi
\item[- ] gửi dữ liệu đến đối tác của chúng.
\item[- ] nhận dữ liệu từ đối tác của chúng.
\item[- ] Nếu rank của chúng nhỏ, thì giữ lại các phần tử nhỏ nhất(dữ liệu của bộ xử lí + dư liệu của đối tác).
\item[- ] Trái lại thì giữ lại các phần tử lớn nhất(dữ liệu của bộ xử lí + dữ liệu của đối tác).
\ei
\ei
\ei
\end{frame}
\section{Kết quả}
\subsection{Thực hiện}
\subsection{Kết quả}
% TODO: Book
\begin{frame}{Tài liệu tham khảo}
\section*{Tài liệu tham khảo}

\end{frame}
\end{document}